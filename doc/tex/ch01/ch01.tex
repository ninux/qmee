\chapter{Introduction}

\section{Exercise 1-1}
The equation for the momentum
\begin{equation}
	p = \frac{h}{\lambda}
\end{equation}
has the following dimensions:
\[
	\left[\text{momentum}\right] = \frac{\left[\text{energy}\right] \left[\text{time}\right]}{\left[\text{length}\right]}
\]
The unit of energy is \si{\joule}, the unit of time is \si{\second}, and the
unit of length is \si{\meter}. Hence, we can perform the dimensional analysis
directly with the corresponding units which gives
\[
	p
	= \left[ \frac{\si{\joule} \si{\second}}{\si{\meter}} \right]
	= \left[ \frac{\si{\kilogram\meter\squared\second}}{\si{\second\squared\meter}} \right]
	= \left[ \frac{\si{\kilogram\meter}}{\si{\second}} \right]
\]

\section{Exercise 1-2}
The workfunction $\phi$ of Titanium is
\[
	\phi
	= \SI{4.33}{\eV}
	= \SI{4.33}{\eV} \frac{\SI{1.6e-19}{\joule}}{\SI{1}{\eV}}
	= \SI{6.93e-19}{\joule}
\]
The work function $\phi$ describes the amount of energy that is required to
free an electron by incident EM radiation (i.e. light). Hence, this is the
absolute minimum required energy. Using the generic relation
$E - \phi = h f$ we thus get $E - \phi = 0$ and thus $E = \phi$. As we know
the energy and Planck's constant, we can solve for the frequency $f$
\[
	E = h f \quad \Rightarrow \quad f = \frac{E}{h}
\]
However, we are interested in the required frequency of the incident EM
radiation. Hence, we use the relation $c_0 = \lambda f$. With this we get
\[
	\lambda
	= \frac{c_0 h}{E}
	= \frac{\SI{3e8}{\meter\per\second} \cdot \SI{6.625e-34}{\joule\second}}{\SI{6.93e-19}{\joule}}
	= \SI{2.87e-7}{\meter}
\]

\lstinputlisting[caption=Exercise 1-2]{./../ch1/ex_1_2.m}

\section{Exercise 1-3}
At the beginning, the electron is considered to have only KE and no PE. The
amount of initial KE ($KE_1$) can be calculated with the given initial
wavelength of $\lambda_1 = \SI{10}{\nano\meter}$. This gives us
\[
	KE_1
	= \frac{1}{2 m_e} \left(\frac{h}{\lambda_1}\right)^2
	= \frac{1}{2 \SI{9.11e-31}{\kg}} \left(\frac{\SI{6.625e-34}{\joule\second}}{\SI{10}{\nano\meter}}\right)^2
	= \SI{2.41e-21}{\joule}
\]
After going trough the potential of \SI{0.02}{\eV}, the electron gains
another \SI{0.02}{\eV} worth of KE. Hence, the resulting KE is
\[
	KE_2
	= KE_1 + \SI{0.02}{\eV}
	= \SI{2.41e-21}{\joule} + \SI{0.02}{\eV}\left(\frac{\SI{1.6e-9}{\joule}}{\SI{1}{\eV}}\right)
	= \SI{5.61}{\joule}
\]
Now that the resulting KE is known, the wavelength of the electron can be
determined as
\[
	\lambda
	= \frac{h}{\sqrt{2 KE_2 m_e}}
	= \frac{\SI{6.625e-34}{\joule\second}}{\sqrt{2 \SI{5.61}{\joule} \SI{9.11e-31}{\kg}}}
\]

\lstinputlisting[caption=Exercise 1-3]{./../ch1/ex_1_3.m}

\section{Exercise 2-1}
As the particle goes into the barrier, a certain portion of its inital KE
is exchanged for PE. While the total energy stays the same, the wavelength
changes as it depends on the KE of the particle. As the particle has lower
KE, the wavelenght gets larger (i.e. the frequency geth lower) according to
\[
	KE = \frac{1}{2 m_e} \left(\frac{h}{\lambda}\right)^2
	\Rightarrow
	\lambda = \frac{h}{\sqrt{2 KE m_e}}
\]

The imaginary part \emph{leads} the real part. This means that the direction
in which the wave propagates can be determined from the plot. The incident
wave prior to the particle arriving at the barrier, has a leading imaginary
part to the right. Hence, the particle is moving to the right and thus
towards the potential barrier. After the barrier is reached, the wave is
split in two parts; the incident wave is split in a transmissive and
reflective part. The transmissive part is the part riht of the barrier, which
is in the direction of the incident wave. This is evident by the leading
imaginary part of the wave. However, the reflected part shows an imaginary
part which is to the left of the real part. Hence, this reflected part is
propagating in opposite direction of the incident wave.

\section{Exercise 2-2}
In general, from an electrical engineers point of viwe, the Heisenberg
uncertainty principle describes the same relation as signals in the time and
frequency domain when performing the Fourier transforms. If a signal is very
localized in time (e.g. dirac delta function or step function), it's
bandwidth is very large (e.g. towards infinite bandwidth or white noise).
The same holds for the opposite direction. If a signal is very localized in
the frequency domain (i.e. a perfectly constant sine wave of frequency $f$),
the signal is very wide-spread in the time domain (i.e. starts in
$-\infty$ and goes until $+\infty$).

Figure 1.5 shows the wave of a particle which spreads out spatially about
\SI{20}{\nano\meter}. If we want to know the position of the particle
described by the shown wave in Figure 1.5, we can not perfectly localize
it. Hence, we have an uncertainty in position of the particle.

We can see that the center wavelenth is approximately \SI{10}{\nano\meter}.
However, the wave is not a simple sine wave consisting of a signle frequency
$f$. In fact, the shown waveform is a Gaussian envelop and this means that
the shown waveform consists of many different frequencies or wavelengths
around the center wavelength of \SI{10}{\nano\meter}. The momentum of the
particle is described by
\[
	p = \frac{h}{\lambda}
\]
If we want to know the exact momentum of the particle, we need an exact
value for its wavelength $\lambda$. However, herein lied the difficulty,
as we do not knwo the exact wavelength of the particle and hence we
have an uncertainty in the particles momentum.

In Figure 1.5 we see a typical example of a particle, that is we have an
uncertainty in both position and momentum. However, it would be good to
imagine the two extreme cases where the uncertainty of either the momentum
or the position is going towards zero.

\section{Exercise 2-3}

\subsection{1-D Case}
To derive the unit of the 1-D wavefunction $\psi(x)$ we can analyze the
normalized probability calculation for the position
\[
	\braket{\psi(x)|\psi(x)}
	= \int_{-\infty}^{+\infty}{\abs{\psi(x)}^2} \dif x
	= 1
\]
The physical interpretation of this equaiton is that the square of the 
wavefunction gives the probability(density) to find the particle in a
region of \emph{distance} $x$. If we integrate this along $x$, we get
the probability that the particle is within the investigated region of $x$.

In this equation we see that we integrate over distance ($\dif x$). Hence,
the integration gives a dimension of length [L]. As the result is $1$,
which is dimensionless, the integrand must be of the inverse
unit, i.e. $[L^{-1}]$. The integrand is the square of the wavefunction
$\abs{\psi(x)}^2$. Therefore, the wavefunction itself must have the unit of
$[\sqrt{L^{-1}}]$ or $[L^{\frac{-1}{2}}]$.

\subsection{2-D Case}
In the 2-D case, we no longer investige a simple distance but a \emph{surface}.
Thus, instead of $\int \dif x$ we now use $\int \dif s$. Analogous to the
1-D case, we can identify that the integration gives a dimension of $[L^2]$.
To get a dimensionless result, the square of the wavefunction
$\abs{\psi(x,y)}^2$ must have the unit of $[L^{-2}$. Therefore, the
wavefunction itself must have the unit of $[\sqrt{L^{-2]}}]$ or $[L^{-1}]$.

\subsection{3-D Case}
Analogous to the 3-D case, we investigate a \emph{volume}. Hence, we now
use $\int \dif V$. The integration thus gives a dimension of $[L^3]$. To
get a dimensionless result, the square of the wavefunction
$\abs{\psi(x,y,z)}^2$ must have the unit of $[L^{\frac{-3}{2}}]$.

% \begin{table}[h!]
% 	\centering
% 	\caption{Dimension and units of the wavefunction}
% 	\begin{tabular}{l c c}
% 		\toprule
% 		dimension 	& dimension of $\psi$	& unit				\\
% 		\midrule
% 		\bottomrule
% 	\end{tabular}
% \end{table}
