\chapter{Introduction}

\section{Exercise 1-1}
The equation for the momentum
\begin{equation}
	p = \frac{h}{\lambda}
\end{equation}
has the following dimensions:
\[
	\left[\text{momentum}\right] = \frac{\left[\text{energy}\right] \left[\text{time}\right]}{\left[\text{length}\right]}
\]
The unit of energy is \si{\joule}, the unit of time is \si{\second}, and the unit of length is \si{\meter}. Hence, we can perform the dimensional analysis directly with the corresponding units which gives
\[
	p
	= \left[ \frac{\si{\joule} \si{\second}}{\si{\meter}} \right]
	= \left[ \frac{\si{\kilogram\meter\squared\second}}{\si{\second\squared\meter}} \right]
	= \left[ \frac{\si{\kilogram\meter}}{\si{\second}} \right]
\]

\section{Exercise 1-2}
The workfunction $\phi$ of Titanium is
\[
	\phi
	= \SI{4.33}{\eV}
	= 4.33 \frac{\SI{1.6e-19}{\joule}}{\SI{1}{\eV}}
	= \SI{6.93e-19}{\joule}
\]
The work function $\phi$ describes the amount of energy that is required to free an electron by incident EM radiation (i.e. light). Hence, this is the absolute minimum required energy. Using the generic relation $E - \phi = h f$ we thus get $E - \phi = 0$ and thus $E = \phi$. As we know the energy and Planck's constant, we can solve for the frequency $f$
\[
	E = h f \quad \Rightarrow \quad f = \frac{E}{h}
\]
However, we are interested in the required frequency of the incident EM radiation. Hence, we use the relation $c_0 = \lambda f$. With this we get
\[
	\lambda
	= \frac{c_0 h}{E}
	= \frac{\SI{3e8}{\meter\per\second} \cdot \SI{6.625e-34}{\joule\second}}{\SI{6.93e-19}{\joule}}
	= \SI{2.87e-7}{\meter}
\]

\lstinputlisting[caption=Exercise 1-1]{./../ch1/ex_1_1.m}
