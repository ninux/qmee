\chapter{Stationary States}

\newpage
\section{Exercise 2-1-1}
The eigenvalues are defined as
\[
	E = \epsilon_n = \frac{\hbar^2 \pi^2}{2 m_e a^2} n^2
\]
Having a size $a_1 = \SI{10}{\angstrom}$ and $a_2 = \SI{100}{\angstrom}$,
we can evaluate the eigenvalues and compare them. For this comparison, we
can reduce the discussion to the ground state $n = 1$ and further, we can
define that all other variables are the same. Hence, we get
\[
	E = \epsilon \underbrace{\frac{\hbar^2 \pi^2}{2 m_e}}_{q} \frac{1}{a^2}
\]
This shows that the eigenvalue changes with $a^{-2}$. Thus, a well that is
ten times smaller has energy levels higher by a factor of 100.

\lstinputlisting[caption=Exercise 2-1]{./../ch2/ex_2_1.m}

\newpage
\section{Exercise 2-1-2}
For the infinite well, the $\psi$ has strict boundary conditions with
$\psi(0) = \psi(a) = 0$. For the finite well, however, this does no longer
apply and the waveforms \emph{bleeds} into the well with higher energy level.
This means that there is a non-zero probability that the particle is located
outside the well. For the lower energy level, the situation is quite similar.
There is however a major difference, which is that there will be no states
above \SI{3}{\eV}.

\newpage
\section{Exercise 2-1-3}
The probability density of the particle location is given by $\abs{\psi(x)}^2$
and the probability for the particle being located in interval $[x1, \,x2]$
is hence given by the integral
\[
	P = \int_{x_1}^{x_2} \abs{\psi(x)}^2 \dif x
\]
As we have an infinite well, we know that the boundary conditions are
$\psi(0)=\psi(a)=0$ and hence, only those functions that fulfill these boundary
conditions are valid solutions for the eigenstates. The first eigenstate of the
well has a wavelengths of $\lambda = 2 \times \SI{100}{\angstrom}$, the second
has $\lambda = 2 \times \SI{100}{\angstrom} \div 2$, the third
$\lambda = 2 \times \SI{100}{\angstrom} \div 3$ and so on. Figure
\ref{fig:ch02_ex_2_3.png} shows both the $\psi(x)$ and the probability
density $\abs{\psi(x)}^2$. From the figure we can see the following peaks:

\begin{table}[h]
	\centering
	\begin{tabular}{l l l}
		\hline
		$n$	& $\lambda$			& peaks							\\
		\hline
		1 	& $2a$ 				& $\frac{1}{2}a$ 				\\
		2 	& $a$ 				& $k \frac{1}{4}a$, $k=1,3$ 	\\
		3 	& $\frac{2}{3}a$  	& $k \frac{1}{6}a$, $k=1,3,5$ 	\\
		4 	& $\frac{1}{2}a$  	& $k \frac{1}{8}a$, $K=1,3,5,7$	\\
		\hline
	\end{tabular}
\end{table}

\begin{figure}
	\centering
	\includegraphics[width=\textwidth]{./../ch2/ex_2_3.png}
	\caption{Eigenstates and probability density for $n={1,2,3,4}$.}
	\label{fig:ch02_ex_2_3.png}
\end{figure}

\section{Exercise 2-1-4}
For $n=1$, the particle is in it's ground state and the eigenvalue is
therefore the kinetic energy. Hence, we have
\[
	KE
	= E
	= \epsilon_n
	= \frac{\hbar^2 \pi^2}{2 m_e a^2} n^2
	= \SI{3.77e-3}{\eV}
\]

\section{Exercise 2-1-5}
The general solution is
\[
	\psi(x) = A \sin\left(\frac{n \pi}{a} x\right)
\]
and we need to normalize it such that
\[
	\int_{-\infty}^{+\infty} \psi^{\ast}(x) \psi(x) \dif x = 1
\]
From the normalization we know that
\[
	1 = \frac{A^{\ast}A}{2} a
\]
We can rearrange it for A and get
\[
	A^\ast A = \frac{2}{a}
\]
As we still do not know what A is specifically, we can substitute it with a generic complex form as
\[
	A = r e^{j \varphi}
\]
and hence, we get
\[
	r e^{j \varphi} \, r e^{- j \varphi} = \frac{2}{a}
\]
From the above equation we see that
\[
	r^2 = \frac{2}{a} \quad \Rightarrow \quad r = \sqrt{\frac{2}{a}}
\]
while $\phi$ can be any angle as they cancel out. So, the general solution for $A$ is
\[
	A = \sqrt{\frac{2}{a}} e^{j \varphi} \quad, \mathrm{arbitrary } \, \varphi
\]